\documentclass[runningheads,a4paper]{llncs}
%
\usepackage{natbib} % bibliography stuff
%
\usepackage{graphicx} % allows for working with images
\DeclareGraphicsExtensions{.pdf,.png,.jpeg} % configures latex to look for the following image extensions
%
\usepackage{setspace} % allows for configuring the linespacing in the document
%\singlespacing
\onehalfspacing
%\doublespacing
%
\usepackage{appendix}
%
\usepackage{amssymb}
\setcounter{tocdepth}{3}

\usepackage{url}
\urldef{\mailsa}\path|dkirwan@tssg.org|
\newcommand{\keywords}[1]{\par\addvspace\baselineskip
\noindent\keywordname\enspace\ignorespaces#1}

\begin{document}
\mainmatter  % start of an individual contribution

% first the title is needed
\title{Amateur Radio Astronomy:\\
with Software Defined Radio}

% a short form should be given in case it is too long for the running head
\titlerunning{Amateur Radio Astronomy with Software Defined Radio}

% the name(s) of the author(s) follow(s) next
%
% NB: Chinese authors should write their first names(s) in front of
% their surnames. This ensures that the names appear correctly in
% the running heads and the author index.
%
\author{David Kirwan%
%\thanks{Please note that the LNCS Editorial assumes that all authors have used
%the western naming convention, with given names preceding surnames. This determines
%the structure of the names in the running heads and the author index.}%
\and Alan Davy\and John Ronan}
%
\authorrunning{Amateur Radio Astronomy with Software Defined Radio}
% (feature abused for this document to repeat the title also on left hand pages)

% the affiliations are given next; don't give your e-mail address
% unless you accept that it will be published
\institute{Waterford Institute of Technology,\\Dept of Maths and Physics,\\
Cork Rd, Waterford City, Ireland\\
\mailsa\\
\url{http://www.wit.ie}}

%
% NB: a more complex sample for affiliations and the mapping to the
% corresponding authors can be found in the file "llncs.dem"
% (search for the string "\mainmatter" where a contribution starts).
% "llncs.dem" accompanies the document class "llncs.cls".
%

\toctitle{Amateur Radio Astronomy with Software Defined Radio}
\tocauthor{Amateur Radio Astronomy with Software Defined Radio}
\maketitle


\begin{abstract}
The abstract should summarize the contents of the paper and should
contain at least 70 and at most 150 words. It should be written using the
\emph{abstract} environment.
\keywords{radio astronomy, software defined radio, signal processing}
\end{abstract}

%
%\chapter{Introduction}
%\addcontentsline{toc}{chapter}{Introduction}
%
\section{Introduction}
%\addcontentsline{toc}{section}{Introduction}

\subsection{Background}
%\addcontentsline{toc}{subsection}{Background}
This is typically an outline description detailing the background	 to the problem.
  
%
\newpage
%
\subsection{Scope}
%\addcontentsline{toc}{subsection}{Scope}
Students should identify whether the research outcomes are likely to have universal application or have a defined scope. This is oimprotant in guaging the extent to which the work is capable of independent replication.

%
\newpage
%
\subsection{Research Questions}
%\addcontentsline{toc}{subsection}{Research Questions}
A clear, precise definition of the problem is very important to focus on the research activity. great care should be used in devising the research questions. They define the struture of the investigation/innovation that will be used and an essential metric of the quality of the dissertation is the degree to which the research question has/have been answered.

%
\newpage
%
\subsection{Methodology}
%\addcontentsline{toc}{subsection}{Methodology}
This should outline the approach and methodology being proposed by the student to address the research question.

%
\newpage
%
\subsection{Preliminary Literature Review}
%\addcontentsline{toc}{subsection}{Preliminary Literature Review}
This should contain a review of a number of books, journal articles and web references of relevance to the research area proposed. The literature should contain seminal and recent referenced research material that is categorised under a number of relevant sub-themes.

%
\newpage
%
\subsection{Contribution to Research Knowledge Anticipated}
%\addcontentsline{toc}{subsection}{Contribution to Research Knowledge Anticipated}
%
A dissertation is a work of scholarly investigation that is grounded in the research literature and differs from a report or a book. It is judged on a prescribed set of academic criteria. Although the likely outcomes are tentative at hte start of the program, it is useful to incorporate them into the research proposal to help focus the work program.

%
\newpage
%
\subsection{Description of the Experimental Design / Validation Methodology}
%
A dissertation must employ rigorous scientific argument. The experimental design and the validation methodology must be specified in great detail in the proposal. At this proposal stage you should define clear evaluation criteria.

%
\newpage
%
\subsection{Special Resources Required}
%
The research work may require access to specialised equipment, software, journals and so on.

%
\newpage
%
\subsection{Main Milestones Anticipated}
%
Students should agree a number of milestones and their likely delivery dates with their supervisor at the start of the progress.

%
\newpage
%
\bibliographystyle{plainnat}
\bibliography{bibliography/bibtex}
%
\newpage
%
\appendix
\section{Appendix}
Here is some content in the appendix

\begin{subappendices}
\subsection{How I became inspired}
Lorem ipsum dolor sit amet, consectetur adipiscing elit. Praesent ut egestas sapien. Sed vehicula, libero vitae ornare interdum, nunc felis rhoncus risus, ut lobortis quam ligula sed nunc. Suspendisse potenti. Proin lacinia ex dui, eu maximus justo consequat porttitor. Pellentesque sollicitudin rutrum ex hendrerit vestibulum. Etiam luctus leo vitae magna sagittis feugiat a vitae ligula. Maecenas suscipit interdum tincidunt. Etiam a sapien elit. Nam dictum sed felis non commodo.

%
\end{subappendices}

\end{document}


